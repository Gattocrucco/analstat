% Giacomo Petrillo

\chapter*{Introduzione}

Questo documento è una trascrizione, fedele il giusto,
delle lezioni del corso \emph{Analisi statistica dei dati}
tenuto al dipartimento di Fisica dell'università di Pisa
l'anno accademico 2017--2018,
dai professori Giovanni Punzi, Michael J. Morello e Paolo Francavilla.
Mi sono basato quasi totalmente sui miei appunti,
con queste eccezioni:
\begin{itemize}
	\item sono mancato alla lezione del 25 ottobre\footnote{Era alle 9:00 di mattina.} che ho trascritto dagli appunti di Francesco Serra\footnote{Ma ovviamente mi assumo la responsabilità di eventuali errori.};
	\item la lezione del 15 dicembre l'ho in parte studiata dall'articolo lì citato.
\end{itemize}
Le principali mancanze sono:
\begin{itemize}
	\item la prima lezione;
	\item la lezione del 15 novembre (proprietà asintotiche di cumulante e quantili);
	\item la lezione del 10 novembre sulle funzioni psicometriche,
	di cui ho solo messo il riferimento all'articolo. 
\end{itemize}
Il sorgente di questo documento è reperibile su github:\\
\url{https://github.com/Gattocrucco/analstat}.
