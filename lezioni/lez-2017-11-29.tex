% Giacomo Petrillo
% lezione di Francavilla

\begin{exercise}
	Applicare likelihood ratio all'esponenziale.
\end{exercise}

\begin{solution*}
	Scriviamo l'esponenziale in funzione della media:
	\begin{equation*}
		p(t;\tau)
		= \frac1\tau e^{-t/\tau}.
	\end{equation*}
	La funzione di ordinamento che usiamo è
	\begin{align*}
		R_\tau(t)
		&\is \log \frac {p(t;\tau)} {\sup\limits_{\tau'} p(t;\tau')} = \\
		\intertext{(la stima di massima likelihood è $\tau=t$)}
		&= \log \frac {\frac1\tau e^{-t/\tau}} {\frac1t e^{-1}} = \\
		&= \log\frac t\tau + 1 - \frac t\tau.
	\end{align*}
	La funzione $\log x - x$ è convessa,
	quindi la regione $R_\tau(t)>R_\mathrm{min}$ è un intervallo,
	che scriviamo come $(q_\mathrm{min}\tau, q_\mathrm{max}\tau)$.
	Scriviamo l'equazione del coverage:
	\begin{align*}
		\mathrm{CL}
		&= \int_{q_\mathrm{min}\tau}^{q_\mathrm{max}\tau} \frac{\de t}\tau e^{-t/\tau} = \\
		&= e^{-q_\mathrm{min}} - e^{-q_\mathrm{max}}.
	\end{align*}
	Ricordando che vale
	\begin{align*}
		R_\mathrm{min} = R_\tau(q_\mathrm{min}\tau)
		&= R_\tau(q_\mathrm{max}\tau) \implies \\
		\implies q_\mathrm{min} e^{1-q_\mathrm{min}}
		&= q_\mathrm{max} e^{1-q_\mathrm{max}},
	\end{align*}
	abbiamo due equazioni per $q_\mathrm{min}$ e $q_\mathrm{max}$,
	che vanno risolte numericamente.
	Notiamo che abbiamo ottenuto le sezioni lungo $t$ della banda di confidenza
	indipendentemente da $\tau$
	usando implicitamente il pivot $t/\tau$.
	La stima intervallare è infine
	\begin{align*}
		\tau_\mathrm{min}
		&= \frac t {q_\mathrm{max}}, \\
		\tau_\mathrm{max}
		&= \frac t {q_\mathrm{min}}.
	\end{align*}
\end{solution*}
