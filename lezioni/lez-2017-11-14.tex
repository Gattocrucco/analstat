% Giacomo Petrillo
% lezione di Punzi

Quando, riportando il risultato di un'inferenza, si scrive
\begin{equation*}
	\text{<<}\theta = a \pm b\text{>>},
\end{equation*}
convenzionalmente si intende che $a$ è una stima di $\theta$ con varianza che vale $b^2$ se calcolata in $a$,
e bias trascurabile rispetto a $b$.
Tipicamente si da anche per scontato che,
se non specificato esplicitamente,
la distribuzione dello stimatore sia approssimabile con una gaussiana.
A volte si riporta un terzo numero:
\begin{equation*}
	\text{<<}\theta = a \pm b \pm c\text{>>},
\end{equation*}
dove $c$ quantifica in qualche modo l'incertezza sistematica.
A volte si trovano incertezze \emph{asimmetriche}
\begin{equation*}
	\text{<<}\theta = a^{+b}_{-c}\text{>>}
\end{equation*}
che per adesso lasciamo da parte.

\begin{example}
	Consideriamo la poissoniana.
	Vogliamo stimare la media $\mu$ da un conteggio $k$.
	Sappiamo che il conteggio stesso è lo stimatore di massima likelihood,
	ha bias nullo ed è sufficiente quindi anche efficiente,
	allora lo usiamo senza indugio.
	La varianza di $k$ è $\mu$,
	quindi scriveremo
	\begin{equation*}
		\text{<<}\mu = k \pm \sqrt k\text{>>}.
	\end{equation*}
	Notiamo che, se $k=0$, otteniamo
	\begin{equation*}
		\text{<<}\mu = 0 \pm 0\text{>>};
	\end{equation*}
	questo modo di scrivere il risultato non ci piace perché sembra dire che $\mu$ è sicuramente nullo.
	In generale quello che vorremmo è una stima della varianza con una certa incertezza relativa,
	perché è uso riportare l'incertezza con un certo numero di cifre significative.
	L'incertezza relativa sulla stima della varianza è $\sqrt\mu / k$,
	quindi per $\mu$ piccolo (e quindi $k$ piccoli) la notazione ha un significato molto meno forte.
\end{example}

Ancor più in generale,
quando la deviazione stadard dello stimatore varia sensibilmente
(ad esempio, se vogliamo riportare l'incertezza con due cifre, più dell'\SI1\%)
in un'intorno del valore vero di larghezza dell'ordine della deviazione standard stessa,
riportare la deviazione standard calcolata nel valore dello stimatore ottenuto è poco utile.
Sono stati studiati metodi ad hoc per riportare l'incertezza in questi casi,
ma non sono definitivamente soddisfacenti e comunque non sono applicabili in generale.
Passando al primo ordine, vorremmo che:
\begin{align*}
	\dv{\sigma_{\hat\theta}}{\theta} \cdot \sigma_{\hat\theta}
	&\ll \sigma_{\hat\theta} \implies \\
	\implies \dv{\sigma_{\hat\theta}}{\theta}
	&\ll 1.
\end{align*}
Un caso specifico in cui bisogna conoscere con precisione la varianza è la media pesata\footnote{Vedi \autoref{th:wavg}.}:
in questo caso la varianza entra direttamente del valore del nuovo stimatore.

C'è un altro problema con questa notazione,
in questo caso non intrinseco ma di errata interpretazione.
Al risultato
<<$\theta=a\pm b$>>
non è in generale associato in alcun modo l'intervallo $(a-b,a+b)$.
Purtroppo è d'uso riportare l'incertezza nei grafici proprio con una ``barra d'errore''
che raffigura questo intervallo.
