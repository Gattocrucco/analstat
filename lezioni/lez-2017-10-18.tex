% Giacomo Petrillo
% lezione di Francavilla

\begin{exercise}
	\label{th:troncexp}
	Consideriamo un'esponenziale troncata a destra:
	\begin{equation*}
		p(t;\lambda)
		= \frac{\lambda e^{-\lambda t}}{1 - e^{-\lambda T}},
		\quad t \in (0,T).
	\end{equation*}
	Calcolare l'informazione di Fisher per $\lambda$.
\end{exercise}

\begin{solution*}
	\begin{align*}
		\log p(t;\lambda)
		&= \log\lambda - \lambda t - \log(1-e^{-\lambda T}) \\
		\frac{\partial}{\partial\lambda} \log p(t;\lambda)
		&= \frac1\lambda - t - \frac{-(-T)e^{-\lambda T}}{1-e^{-\lambda T}} = \\
		&= \frac1\lambda - t - \frac{T}{e^{\lambda T}-1} \\
		-\frac{\partial^2}{\partial\lambda^2} \log p(t;\lambda)
		&= \frac1{\lambda^2} - \frac{T^2e^{\lambda T}}{(e^{\lambda T}-1)^2}.
	\end{align*}
	Notiamo che per $\lambda T\to0$ l'informazione va a zero mentre per $\lambda T\to\infty$
	riotteniamo l'informazione dell'esponenziale.
	Calcoliamo la variazione relativa rispetto a quest'ultima:
	\begin{equation*}
		\frac{I_T(\lambda)}{I_\infty(\lambda)}
		= \lambda^2 I_T(\lambda)
		= 1 - \frac {(\lambda T)^2e^{\lambda T}} {(e^{\lambda T}-1)^2}.
	\end{equation*}
	Ad esempio, per $\lambda T=5$ l'informazione cala del \SI{17}\percent.
	Se immaginiamo di perdere bernouillianamente una frazione
	\begin{equation*}
		\epsilon
		= \int_T^\infty \de t\, \lambda e^{-\lambda t}
		= e^{-\lambda T}
	\end{equation*}
	delle misure di un'esponenziale,
	la variazione relativa dell'informazione di Fisher è~$-\epsilon$
	che per $\lambda T=5$ vale \SI{0.7}\percent.
	\marginpar{Questa cosa non è chiarissima perché se immagino di misurare il tempo tra un evento e l'altro e non osservo un evento con probabilità $\epsilon$ (vedi \autoref{th:salvexp}) l'informazione di Fisher non cambia.
	Quello che qui si intende è come se facessi le misure e poi ne buttassi via alcune (tipo trigger).}
\end{solution*}

\begin{exercise}
	Calcolare l'informazione di Fisher di un'esponenziale troncata a sinistra.
\end{exercise}

\begin{solution}
	Scriviamo la pdf:
	\begin{align*}
		p(t;\lambda)
		&= \frac {\lambda e^{-\lambda t}} {\int_T^\infty \de t\, \lambda e^{-\lambda t}} = \\
		&= \lambda e^{-\lambda (t-T)}, \quad t\in(T,\infty).
	\end{align*}
	Com'era intuitivo è un'esponenziale traslata quindi l'informazione di Fisher è ancora~$1/\lambda^2$.
\end{solution}


