%%%% questa parte l'ho scritta il 2022-01-07
%%%% la lezione originale era diversa, aveva fatto proprietà asintotiche
%%%% dei quantili, ma non avevo dei buoni appunti quindi non l'avevo
%%%% trascritta

\subsection{Intervalli interquantile}

Prima di studiare la stima intervallare in tutta la sua gloria, vediamo ancora
un altro modo semplice di costruire intervalli a partire da una distribuzione.

\begin{definition}[Quantile]
    %
    Sia $F$ la cdf di una variabile a valori reali. Il \emph{quantile} di
    ordine $\alpha$ è $q_\alpha \is F^{-1}(\alpha)$.
    %
\end{definition}%
%
In altre parole il quantile di ordine $\alpha$ è il valore di $x$ tale che
la probabilità che $x$ sia minore è $\alpha$. In particolare $q_{1/2}$ è la
mediana.

\begin{exercise}
    %
    Se trasformo $x$ con una funzione $\fundef{\mathbb R}{\mathbb R}$, come
    trasformano i quantili?
    %
\end{exercise}

\begin{solution}
    %
    Sia $y = f(x)$. Se $f$ è strettamente crescente, abbiamo che
    %
    \begin{equation*}
        %
        \alpha = P(x < q^x_\alpha)
        = P(f(x) < f(q^x_\alpha))
        = P(y < f(q^x_\alpha))
        \implies q^y_\alpha = f(q^x_\alpha).
        %
    \end{equation*}
    %
    Se $f$ è strettamente descrescente, i quantili si invertono:
    %
    \begin{align*}
        %
        1 - \alpha &= P(x < q^x_{1-\alpha})
        = P(f(x) > f(q^x_{1-\alpha}))
        = 1 - P(y < f(q^x_{1-\alpha}))
        \implies \\
        %
        \implies q^y_\alpha &= f(q^x_{1-\alpha}).
        %
    \end{align*}
    %
    Invece se $f$ non è biunivoca, o se comunque ``rimischia'' il dominio
    di $x$, in generale i quantili possono cambiare arbitrariamente.
    %
\end{solution}

\begin{definition}[Intervallo interquantile]
    %
    Un \emph{intervallo interquantile $\alpha$-$\beta$}, con $0 \le \alpha <
    \beta \le 1$, è l'intervallo delimitato dai quantili di ordine $\alpha$
    e $\beta$, $(q_\alpha, q_\beta)$.
    %
\end{definition}%
%
Ne segue che la probabilità che $x$ cada nell'intervallo $\alpha$-$\beta$ è
%
\begin{equation*}
    %
    P(q_\alpha < x < q_\beta)
    = P(x < q_\beta) - P(x < q_\alpha)
    = \beta - \alpha.
    %
\end{equation*}
%
Se si prende $\alpha = 0$, l'intervallo arriva fino al bordo sinistro del
supporto della distribuzione di $x$; analogamente se $\beta = 1$ arriva fino
al bordo destro.

Se si ha un campione di dati $x_0, \ldots, x_{N-1}$, i quantili si possono
stimare in questo modo: si mettono in ordine crescente le $x_i$, e si fa la
proporzione $\alpha:i = 1:(N-1)$, ottenendo $i = \alpha (N-1)$. In questo modo
$0 \le i \le N-1$. Si arrotonda $i$ e si prende l'$x_i$ corrispondente. Di
conseguenza circa una frazione $\alpha$ dei dati sta a sinistra dell'$x_i$
scelto.

L'intervallo interquantile è comodo per riassumere con delle barre d'errore su
un grafico le distribuzioni di vari campioni. Calcolare la deviazione standard
funziona bene quando la distribuzione è circa gaussiana, ma se ci sono
distribuzioni diverse, magari diverse per ogni barra da mostrare, la deviazione
standard può essere fuorviante. Per la gaussiana l'intervallo
$(-\sigma,\sigma)$ ha probabilità circa 68\,\%. Per mostrare delle barre
d'errore che nel caso della gaussiana corrispondano a $\pm1\sigma$, si può
allora usare l'intervallo interquantile 16--84\,\%.
