% Giacomo Petrillo
% lezione di Morello

\begin{definition}[Probabilità]
	Definiamo in astratto la probabilità $\fundef[P]{\pset(S)}\R$ come funzione dai sottoinsiemi di un insieme universo $S$ a valori reali, che soddisfa i seguenti \emph{assiomi di Kolmogorov}:
	\begin{enumerate}
		\item $\forall A\subseteq S:P(A) \ge 0$;
		\item $\forall A,B\subseteq S: A\cap B=\emptyset \rightarrow P(A)+P(B)=P(A\cup B)$; \label{kolm2}
		\item $P(S)=1$.
	\end{enumerate}
\end{definition}

\begin{theorem}[Probabilità dell'unione]
	$P(A\cup B) = P(A) + P(B) - P(A\cap B)$
\end{theorem}

\begin{proof}
	Scrivo $B$ come l'unione di due insiemi disgiunti
	\[B = (A\cap B) \cup (\comp A \cap B)\]
	dove $\comp A\is S\setminus A$; per l'assioma \ref{kolm2} segue
	\[P(B) = P(A\cap B) + P(\comp A \cap B).\]
	Scrivo $A\cup B$ come l'intersezione di se stesso con $S=A\cup\comp A$
	\[A\cup B = (A\cup\comp A) \cap (A\cup B)\]
	e applico la proprietà distributiva
	\begin{align*}
		A\cup B
		&= (A\cap A) \cup (A\cap B) \cup (\comp A\cap A) \cup (\comp A \cap B) = \\
		&= A \cup (\comp A\cap B)
	\end{align*}
	che sono disgiunti, da cui segue
	\[P(A\cup B) =  P(A) + P(\comp A\cap B)\]
	sostituendo nell'equazione per $P(B)$ ottengo la tesi.
\end{proof}

\begin{definition}[Probabilità condizionata]
	Sia $P(C)\neq 0$. Definiamo la \emph{probabilità di $A$ dato $C$} come
	\[P(A|C) \is \frac{P(A\cap C)}{P(C)}.\]
\end{definition}

Il senso della probabilità condizionata è che se $C$ diventa certo, devo modificare le probabilità degli altri insiemi in modo che il nuovo insieme universo sia $C$.

\begin{exercise}
	Verificare che gli assiomi di Kolmogorov continuano a valere se, per fissato $C$, sostituisco in generale $P(X)$ con $P(X|C)$.
\end{exercise}

\begin{definition}[Indipendenza]
	$A$ e $B$ sono \emph{indipendenti} se $P(A|B) = P(A)$.
\end{definition}

\begin{exercise}
	Se $A$ e $B$ sono indipendenti allora $P(A\cap B) = P(A)P(B)$.
\end{exercise}

\begin{exercise}
	Siano $P(A)=\frac1{10}$ e $P(B)=1$, $A$ e $B$ sono indipendenti?
\end{exercise}

\begin{solution}
	Sono indipendenti, perché:
	\begin{align*}
		&1 \ge P(A\cup B) \ge P(B) = 1 \\
		\implies &1 = P(A\cup B) = P(A) + P(B) - P(A\cap B) = \frac{11}{10} - P(A\cap B) \\
		\implies &P(A\cap B) = \frac{1}{10} \\
		&P(A|B) = \frac{P(A\cap B)}{P(B)} = \frac{\frac{1}{10}}{1} = P(A)
	\end{align*}
\end{solution}

\begin{exercise}
	Siano $A$ e $B$ mutualmente esclusivi. Dire cosa significa e se allora sono indipendenti.
\end{exercise}

\begin{solution}
	Significa che $A\cap B=\emptyset$ e in generale non implica l'indipendenza, perché si ha sempre $P(A|B)\propto P(A\cap B)=0$.
\end{solution}
