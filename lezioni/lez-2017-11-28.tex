% Giacomo Petrillo
% lezione di Punzi

\section{Test di ipotesi}

Finora abbiamo trattato l'inferenza con stime puntuali e stime intervallari.
Per le stime puntuali, tuttavia,
ci siamo concentrati solo sui casi in cui si può scegliere la stima migliore in base al bias e alla varianza.
Ci sono casi in cui lo spazio del parametro non ha un'operazione di somma
e non si può neanche calcolare il valore atteso di uno stimatore.
Esiste un'altro insieme di metodi generali di ottimizzazione degli stimatori
per i casi in cui c'è un valore privilegiato del parametro che ci interessa confermare o escludere,
questi metodi vanno sotto il nome di \emph{test di ipotesi}.

\begin{example}
	Vogliamo sapere se la media di una poissoniana è nulla.
	Se $k\neq 0$ allora sicuramente $\mu\neq 0$.
	Se $k=0$ cosa possiamo dire?
\end{example}

\begin{example}
	Abbiamo una gaussiana e vogliamo stabilire se la media è $\mu_1$ o~$\mu_2$.
\end{example}

Esponiamo la notazione e le definizioni di base.
Il valore privilegiato del parametro è chiamato \emph{ipotesi nulla} e indicato con $H_0$.
Gli altri valori (le \emph{ipotesi alternative}) vengono indicati con un indice diverso da zero: $H_i$.
Lo stimatore si chiama \emph{test} e propriamente è una stima intervallare
con solo due risultati possibili: l'ipotesi nulla oppure tutto il resto.
Il risultato del test si dice \emph{positivo} se è diverso dall'ipotesi nulla,
\emph{negativo} altrimenti.

\begin{definition}[Falso positivo]
	Si chiama \emph{falso positivo} o \emph{errore di tipo I}
	il caso in cui il valore vero è $H_0$ e il risultato del test è positivo.
	La probabilità di falso positivo, detta \emph{taglia}, si indica con $\alpha$:
	\begin{equation*}
		\alpha
		\is P(\neg H_0;H_0) = 1 - P(H_0;H_0).
	\end{equation*}
    %
\end{definition}

\begin{definition}[Falso negativo]
	Si chiama \emph{falso negativo} o \emph{errore di tipo II}
	il caso in cui il valore vero è diverso da $H_0$ e il risultato del test è negativo.
	La probabilità di falso negativo si indica con $\beta$:
	\begin{equation*}
		\beta_i
		\is P(H_0;H_i), \quad i \neq 0.
	\end{equation*}
	La \emph{potenza} è $1-\beta$.
\end{definition}

\begin{definition}[Consistenza]
	Un test definito per $N$ estrazioni è \emph{consistente}
	se per $N\to\infty$ la potenza tende a~1:
	\begin{equation*}
		\forall i\neq0:
		\lim_{N\to\infty} \beta_i = 0.
	\end{equation*}
\end{definition}

\begin{definition}[Unbiasedness]
	Un test è \emph{unbiased}
	se la potenza è maggiore della taglia:
	\begin{equation*}
		\forall i\neq0:
		1-\beta_i \ge \alpha.
	\end{equation*}
\end{definition}

L'unbiasedness si può riformulare come $P(H_0;H_0)\ge P(H_0;H_i)$,
cioè: nei casi in cui il risultato del test è negativo,
il risultato deve coincidere con la stima di massima likelihood.
Osserviamo anche che,
se lo spazio del parametro è finito,
la consistenza implica la unbiasedness per $N$ abbastanza grande.

\begin{definition}[Regione critica]	
	\marginpar{Fare illustrazione}
	Un qualunque test può essere scomposto come una \emph{statistica del test $t$}
	dallo spazio della variabile a uno \emph{score space} (tipicamente~$\R$ oppure $t$ è l'identità)
	combinata con una statistica $T$ dallo score space al risultato del test.
	I~valori per cui $T$ dà risultato positivo si chiamano \emph{regione critica $C$},
	i valori per cui dà risultato negativo \emph{regione di accettazione}~($\comp C$).
\end{definition}

\subsection{Ipotesi nulla composta}

Si possono anche definire i test di ipotesi per il caso in cui l'ipotesi nulla
non è un singolo valore del parametro, ma un insieme di valori. Sia $\Theta_0
\subseteq \Theta$ l'insieme delle ipotesi nulle, $\Theta_1 = \Theta \setminus
\Theta_0$ l'insieme delle ipotesi alternative, e indichiamo con $T_0$ e $T_1$ i
risultati negativo e positivo del test. In questo caso si definiscono la
funzione errore di tipo I
%
\begin{equation*}
    %
    \alpha(\theta) \is P(T_1; \theta), \quad \theta \in \Theta_0,
    %
\end{equation*}
%
mentre la funzione errore di tipo II è la stessa di prima, però scriviamola
allo stesso modo di $\alpha(\theta)$:
%
\begin{equation*}
    %
    \beta(\theta) \is P(T_0; \theta), \quad \theta \in \Theta_1.
    %
\end{equation*}
%
Queste due funzioni le uniamo in una singola \emph{funzione potenza} su tutto
il dominio del parametro:
%
\begin{equation*}
    %
    \eta(\theta) \is P(T_1; \theta)
    = \begin{cases}
        \alpha(\theta) & \theta \in \Theta_0, \\
        1 - \beta(\theta) & \theta \in \Theta_1.
    \end{cases}
    %
\end{equation*}
%
Definiamo il \emph{livello di significatività $\alpha$} come il massimo errore
di tipo I:
%
\begin{equation*}
    %
    \alpha \is \sup_\theta \alpha(\theta)
    %
\end{equation*}
%
e diciamo che il test ha taglia $\alpha$ solo se $\alpha(\theta) = \alpha$ per
tutti i $\theta \in \Theta_0$. Livello di significatività e taglia sono simili
ai concetti di livello di confidenza e coverage costante delle stime
intervallari.

\subsection{Test UMP}

Vediamo ora qual è il modo convenzionale di confrontare i test.
Consideriamo il caso in cui ci sono solo due ipotesi.
Allora, a parità di taglia,
il test migliore è quello che ha potenza maggiore cioè minimo $\beta$.
Se le ipotesi sono più d'ue,
questo criterio ha senso solo se accade che la potenza è maggiore per ogni ipotesi alternativa.
Se esiste, il test con la potenza maggiore di tutti gli altri per ogni ipotesi alternativa è detto
\emph{UMP} (uniformly most powerful).

\begin{theorem}[Lemma di Neyman-Pearson]
	\label{th:np}
	Per due ipotesi esiste il test UMP
	ed è definito da una regione critica del tipo
	\begin{equation*}
		C = \Setdef[x]{\frac{p(x;H_0)}{p(x;H_1)} < q(\alpha)}.
	\end{equation*}
\end{theorem}

\begin{proof}
	Indichiamo con le variabili non primate il test dell'enunciato
	e con le variabili primate un test qualunque.
	Dobbiamo mostrare che se $\alpha'\le\alpha$ allora $\beta'\ge\beta$,
	ovvero
	\begin{align*}
		P(\comp C';H_1)
		&\ge P(\comp C;H_1) \iff \\
		\iff P(C;H_1)
		&\ge P(C';H_1). \\
		\intertext{Scomponiamo le regioni critiche in intersezione e differenza:}
		P(C)
		&= P(C\cap C') + P(C\cap\comp C'), \\
		P(C')
		&= P(C'\cap C) + P(C'\cap\comp C). \\
		\intertext{Quindi la disequazione da dimostrare si riscrive come:}
		P(C\cap\comp C';H_1)
		&\ge P(C'\cap\comp C;H_1). \\
		\intertext{Mostriamo che il primo termine è maggiore del secondo:}
		P(C\cap\comp C';H_1)
		&= \int_{C\cap\comp C'} \de x\, p(x;H_1) \ge \\
		\intertext{(per definizione di $C$)}
		&\ge \int_{C\cap\comp C'} \de x\, \frac{p(x;H_0)}q = \\
		&= \frac1q P(C\cap\comp C';H_0) \isis @ \\
		0
		&\le \alpha - \alpha' = \\
		&= P(C;H_0) - P(C';H_0) = \\
		&= P(C\cap\comp C';H_0) - P(C'\cap\bar C;H_0) \implies \\
		\implies P(C\cap\comp C';H_0)
		&\ge P(C'\cap\comp C;H_0) \implies \\
		\implies @
		&\ge \frac1q P(C'\cap\comp C;H_0) = \\
		&= \int_{C'\cap\comp C} \de x\, \frac{p(x;H_0)}q \ge \\
		\intertext{(per definizione di $\comp C$)}
		&\ge \int_{C'\cap\comp C} \de x\, p(x;H_1) = \\
		&= P(C'\cap\comp C;H_1). \qedhere
	\end{align*}
\end{proof}

\begin{fact}[Test one-sided UMP]
	\label{th:umpge}
	Per un modello con parametro $\theta\in[\theta_0,\infty)$
	esiste il test UMP per l'ipotesi nulla $\theta_0$
	se e solo se il modello è nella forma
	\begin{equation*}
		p(x;\theta)
		= F(x) G(\theta) \exp\big(A(x) B(\theta)\big)
	\end{equation*}
	\marginpar{Negli appunti ho segnato che ha dimostrato la freccia facile,
	però non ho scritto la dimostrazione e non sono riuscito a riinventarmela.}%
	con $B(\theta)$ monotona.
	Cambiando il segno di $A$, $B$ può sempre essere presa crescente,
	nel qual caso il test è dato dalla regione critica
	\begin{equation*}
		C = \setdef[x]{A(x) \ge q(\alpha)}.
	\end{equation*}
\end{fact}


% % spostata all'inizio della lezione del 1 dicembre
% A parte il \autoref{th:np} e il \autoref{th:umpge},
% non esistono casi generali in cui si conosce il test UMP.
